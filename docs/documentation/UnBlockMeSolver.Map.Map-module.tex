%
% API Documentation for Unblock Me Solver
% Module UnBlockMeSolver.Map.Map
%
% Generated by epydoc 3.0.1
% [Thu Jul 20 18:43:00 2017]
%

%%%%%%%%%%%%%%%%%%%%%%%%%%%%%%%%%%%%%%%%%%%%%%%%%%%%%%%%%%%%%%%%%%%%%%%%%%%
%%                          Module Description                           %%
%%%%%%%%%%%%%%%%%%%%%%%%%%%%%%%%%%%%%%%%%%%%%%%%%%%%%%%%%%%%%%%%%%%%%%%%%%%

    \index{UnBlockMeSolver \textit{(package)}!UnBlockMeSolver.Map \textit{(package)}!UnBlockMeSolver.Map.Map \textit{(module)}|(}
\section{Module UnBlockMeSolver.Map.Map}

    \label{UnBlockMeSolver:Map:Map}

%%%%%%%%%%%%%%%%%%%%%%%%%%%%%%%%%%%%%%%%%%%%%%%%%%%%%%%%%%%%%%%%%%%%%%%%%%%
%%                               Variables                               %%
%%%%%%%%%%%%%%%%%%%%%%%%%%%%%%%%%%%%%%%%%%%%%%%%%%%%%%%%%%%%%%%%%%%%%%%%%%%

  \subsection{Variables}

    \vspace{-1cm}
\hspace{\varindent}\begin{longtable}{|p{\varnamewidth}|p{\vardescrwidth}|l}
\cline{1-2}
\cline{1-2} \centering \textbf{Name} & \centering \textbf{Description}& \\
\cline{1-2}
\endhead\cline{1-2}\multicolumn{3}{r}{\small\textit{continued on next page}}\\\endfoot\cline{1-2}
\endlastfoot\raggedright \_\-\_\-p\-a\-c\-k\-a\-g\-e\-\_\-\_\- & \raggedright \textbf{Value:} 
{\tt \texttt{'}\texttt{UnBlockMeSolver.Map}\texttt{'}}&\\
\cline{1-2}
\end{longtable}


%%%%%%%%%%%%%%%%%%%%%%%%%%%%%%%%%%%%%%%%%%%%%%%%%%%%%%%%%%%%%%%%%%%%%%%%%%%
%%                           Class Description                           %%
%%%%%%%%%%%%%%%%%%%%%%%%%%%%%%%%%%%%%%%%%%%%%%%%%%%%%%%%%%%%%%%%%%%%%%%%%%%

    \index{UnBlockMeSolver \textit{(package)}!UnBlockMeSolver.Map \textit{(package)}!UnBlockMeSolver.Map.Map \textit{(module)}!UnBlockMeSolver.Map.Map.Map \textit{(class)}|(}
\subsection{Class Map}

    \label{UnBlockMeSolver:Map:Map:Map}
\begin{tabular}{cccccc}
% Line for object, linespec=[False]
\multicolumn{2}{r}{\settowidth{\BCL}{object}\multirow{2}{\BCL}{object}}
&&
  \\\cline{3-3}
  &&\multicolumn{1}{c|}{}
&&
  \\
&&\multicolumn{2}{l}{\textbf{UnBlockMeSolver.Map.Map.Map}}
\end{tabular}


%%%%%%%%%%%%%%%%%%%%%%%%%%%%%%%%%%%%%%%%%%%%%%%%%%%%%%%%%%%%%%%%%%%%%%%%%%%
%%                                Methods                                %%
%%%%%%%%%%%%%%%%%%%%%%%%%%%%%%%%%%%%%%%%%%%%%%%%%%%%%%%%%%%%%%%%%%%%%%%%%%%

  \subsubsection{Methods}

    \vspace{0.5ex}

\hspace{.8\funcindent}\begin{boxedminipage}{\funcwidth}

    \raggedright \textbf{\_\_init\_\_}(\textit{self}, \textit{graph}, \textit{delimeter}={\tt \texttt{'}\texttt{{\textbackslash}n}\texttt{'}})

    \vspace{-1.5ex}

    \rule{\textwidth}{0.5\fboxrule}
\setlength{\parskip}{2ex}
    Initialize the class by setting the graph and delimeters variables.

\setlength{\parskip}{1ex}
      \textbf{Parameters}
      \vspace{-1ex}

      \begin{quote}
        \begin{Ventry}{xxxxxxxxx}

          \item[graph]

          Represents the board that the game will be played on

            {\it (type=string {\textbar} MapReader)}

          \item[delimeter]

          how to split the string

            {\it (type=string)}

        \end{Ventry}

      \end{quote}

      Overrides: object.\_\_init\_\_

    \end{boxedminipage}

    \label{UnBlockMeSolver:Map:Map:Map:convertToMap}
    \index{UnBlockMeSolver \textit{(package)}!UnBlockMeSolver.Map \textit{(package)}!UnBlockMeSolver.Map.Map \textit{(module)}!UnBlockMeSolver.Map.Map.Map \textit{(class)}!UnBlockMeSolver.Map.Map.Map.convertToMap \textit{(method)}}

    \vspace{0.5ex}

\hspace{.8\funcindent}\begin{boxedminipage}{\funcwidth}

    \raggedright \textbf{convertToMap}(\textit{self})

    \vspace{-1.5ex}

    \rule{\textwidth}{0.5\fboxrule}
\setlength{\parskip}{2ex}
    Convert the graph into a matrix that can be viewed and modified.

\setlength{\parskip}{1ex}
    \end{boxedminipage}

    \label{UnBlockMeSolver:Map:Map:Map:setUpPieces}
    \index{UnBlockMeSolver \textit{(package)}!UnBlockMeSolver.Map \textit{(package)}!UnBlockMeSolver.Map.Map \textit{(module)}!UnBlockMeSolver.Map.Map.Map \textit{(class)}!UnBlockMeSolver.Map.Map.Map.setUpPieces \textit{(method)}}

    \vspace{0.5ex}

\hspace{.8\funcindent}\begin{boxedminipage}{\funcwidth}

    \raggedright \textbf{setUpPieces}(\textit{self})

    \vspace{-1.5ex}

    \rule{\textwidth}{0.5\fboxrule}
\setlength{\parskip}{2ex}
    Find every piece available in the board and add it to an array to keep 
    track of them for future use.

\setlength{\parskip}{1ex}
    \end{boxedminipage}

    \label{UnBlockMeSolver:Map:Map:Map:setUp}
    \index{UnBlockMeSolver \textit{(package)}!UnBlockMeSolver.Map \textit{(package)}!UnBlockMeSolver.Map.Map \textit{(module)}!UnBlockMeSolver.Map.Map.Map \textit{(class)}!UnBlockMeSolver.Map.Map.Map.setUp \textit{(method)}}

    \vspace{0.5ex}

\hspace{.8\funcindent}\begin{boxedminipage}{\funcwidth}

    \raggedright \textbf{setUp}(\textit{self})

    \vspace{-1.5ex}

    \rule{\textwidth}{0.5\fboxrule}
\setlength{\parskip}{2ex}
    Set up the Map class and set the graph by either getting a a string 
    from either a map reader or string as the graph and breaking if not 
    valid.

\setlength{\parskip}{1ex}
    \end{boxedminipage}

    \label{UnBlockMeSolver:Map:Map:Map:isWallOrGoal}
    \index{UnBlockMeSolver \textit{(package)}!UnBlockMeSolver.Map \textit{(package)}!UnBlockMeSolver.Map.Map \textit{(module)}!UnBlockMeSolver.Map.Map.Map \textit{(class)}!UnBlockMeSolver.Map.Map.Map.isWallOrGoal \textit{(method)}}

    \vspace{0.5ex}

\hspace{.8\funcindent}\begin{boxedminipage}{\funcwidth}

    \raggedright \textbf{isWallOrGoal}(\textit{self}, \textit{point}, \textit{num\_goals\_found})

    \vspace{-1.5ex}

    \rule{\textwidth}{0.5\fboxrule}
\setlength{\parskip}{2ex}
    This will check if the point is a wall or goal and if not return False.
    Additionally, it will update the number of goals that have been found.

\setlength{\parskip}{1ex}
      \textbf{Parameters}
      \vspace{-1ex}

      \begin{quote}
        \begin{Ventry}{xxxxxxxxxxxxxxx}

          \item[point]

          the point being checked in the graph.

            {\it (type=character)}

          \item[num\_goals\_found]

          number of goals in the graph currently found

            {\it (type=integer)}

        \end{Ventry}

      \end{quote}

      \textbf{Return Value}
    \vspace{-1ex}

      \begin{quote}
      if the point is a wall or a goal and number of goals currently found

      {\it (type=boolean, integer)}

      \end{quote}

    \end{boxedminipage}

    \label{UnBlockMeSolver:Map:Map:Map:largeRowValid}
    \index{UnBlockMeSolver \textit{(package)}!UnBlockMeSolver.Map \textit{(package)}!UnBlockMeSolver.Map.Map \textit{(module)}!UnBlockMeSolver.Map.Map.Map \textit{(class)}!UnBlockMeSolver.Map.Map.Map.largeRowValid \textit{(method)}}

    \vspace{0.5ex}

\hspace{.8\funcindent}\begin{boxedminipage}{\funcwidth}

    \raggedright \textbf{largeRowValid}(\textit{self}, \textit{row\_index}, \textit{num\_goals\_found})

    \vspace{-1.5ex}

    \rule{\textwidth}{0.5\fboxrule}
\setlength{\parskip}{2ex}
    This will test on the larger rows to see if it is valid. A larger row 
    indicates that it is either the top or the bottom of the graph.

\setlength{\parskip}{1ex}
      \textbf{Parameters}
      \vspace{-1ex}

      \begin{quote}
        \begin{Ventry}{xxxxxxxxxxxxxxx}

          \item[row\_index]

          index for the row to be checked

            {\it (type=integer)}

          \item[num\_goals\_found]

          the number of goals that have been found

            {\it (type=integer)}

        \end{Ventry}

      \end{quote}

      \textbf{Return Value}
    \vspace{-1ex}

      \begin{quote}
      If a large row provided is valid and number of goals currently found

      {\it (type=boolean, integer)}

      \end{quote}

    \end{boxedminipage}

    \label{UnBlockMeSolver:Map:Map:Map:numColumnsMatch}
    \index{UnBlockMeSolver \textit{(package)}!UnBlockMeSolver.Map \textit{(package)}!UnBlockMeSolver.Map.Map \textit{(module)}!UnBlockMeSolver.Map.Map.Map \textit{(class)}!UnBlockMeSolver.Map.Map.Map.numColumnsMatch \textit{(method)}}

    \vspace{0.5ex}

\hspace{.8\funcindent}\begin{boxedminipage}{\funcwidth}

    \raggedright \textbf{numColumnsMatch}(\textit{self})

    \vspace{-1.5ex}

    \rule{\textwidth}{0.5\fboxrule}
\setlength{\parskip}{2ex}
    Ensure that the number of columns matches for each row.

\setlength{\parskip}{1ex}
      \textbf{Return Value}
    \vspace{-1ex}

      \begin{quote}
      if the number of columns matches for each row

      {\it (type=boolean)}

      \end{quote}

    \end{boxedminipage}

    \label{UnBlockMeSolver:Map:Map:Map:topBottomRowsValid}
    \index{UnBlockMeSolver \textit{(package)}!UnBlockMeSolver.Map \textit{(package)}!UnBlockMeSolver.Map.Map \textit{(module)}!UnBlockMeSolver.Map.Map.Map \textit{(class)}!UnBlockMeSolver.Map.Map.Map.topBottomRowsValid \textit{(method)}}

    \vspace{0.5ex}

\hspace{.8\funcindent}\begin{boxedminipage}{\funcwidth}

    \raggedright \textbf{topBottomRowsValid}(\textit{self}, \textit{num\_goals\_found})

    \vspace{-1.5ex}

    \rule{\textwidth}{0.5\fboxrule}
\setlength{\parskip}{2ex}
    Make sure that both the top and bottom rows are valid according to the 
    topBottomRowsValid function.

\setlength{\parskip}{1ex}
      \textbf{Parameters}
      \vspace{-1ex}

      \begin{quote}
        \begin{Ventry}{xxxxxxxxxxxxxxx}

          \item[num\_goals\_found]

          number of goals presently found

            {\it (type=integer)}

        \end{Ventry}

      \end{quote}

      \textbf{Return Value}
    \vspace{-1ex}

      \begin{quote}
      If the top and bottom rows are valid and number of goals currently 
      found

      {\it (type=boolean, integer)}

      \end{quote}

    \end{boxedminipage}

    \label{UnBlockMeSolver:Map:Map:Map:midRowsValid}
    \index{UnBlockMeSolver \textit{(package)}!UnBlockMeSolver.Map \textit{(package)}!UnBlockMeSolver.Map.Map \textit{(module)}!UnBlockMeSolver.Map.Map.Map \textit{(class)}!UnBlockMeSolver.Map.Map.Map.midRowsValid \textit{(method)}}

    \vspace{0.5ex}

\hspace{.8\funcindent}\begin{boxedminipage}{\funcwidth}

    \raggedright \textbf{midRowsValid}(\textit{self}, \textit{num\_goals\_found})

    \vspace{-1.5ex}

    \rule{\textwidth}{0.5\fboxrule}
\setlength{\parskip}{2ex}
    Ensure that all the middle rows are valid entries with walls or a goal 
    on both sides.

\setlength{\parskip}{1ex}
      \textbf{Parameters}
      \vspace{-1ex}

      \begin{quote}
        \begin{Ventry}{xxxxxxxxxxxxxxx}

          \item[num\_goals\_found]

          number of goals presently found

            {\it (type=integer)}

        \end{Ventry}

      \end{quote}

      \textbf{Return Value}
    \vspace{-1ex}

      \begin{quote}
      if the middle rows are valid and the number of goals currently found

      {\it (type=boolean, integer)}

      \end{quote}

    \end{boxedminipage}

    \label{UnBlockMeSolver:Map:Map:Map:playerFound}
    \index{UnBlockMeSolver \textit{(package)}!UnBlockMeSolver.Map \textit{(package)}!UnBlockMeSolver.Map.Map \textit{(module)}!UnBlockMeSolver.Map.Map.Map \textit{(class)}!UnBlockMeSolver.Map.Map.Map.playerFound \textit{(method)}}

    \vspace{0.5ex}

\hspace{.8\funcindent}\begin{boxedminipage}{\funcwidth}

    \raggedright \textbf{playerFound}(\textit{self})

    \vspace{-1.5ex}

    \rule{\textwidth}{0.5\fboxrule}
\setlength{\parskip}{2ex}
    Check to see if the player, defined in Map.py, is in the map.

\setlength{\parskip}{1ex}
      \textbf{Return Value}
    \vspace{-1ex}

      \begin{quote}
      player found in map

      {\it (type=boolean)}

      \end{quote}

    \end{boxedminipage}

    \label{UnBlockMeSolver:Map:Map:Map:isValid}
    \index{UnBlockMeSolver \textit{(package)}!UnBlockMeSolver.Map \textit{(package)}!UnBlockMeSolver.Map.Map \textit{(module)}!UnBlockMeSolver.Map.Map.Map \textit{(class)}!UnBlockMeSolver.Map.Map.Map.isValid \textit{(method)}}

    \vspace{0.5ex}

\hspace{.8\funcindent}\begin{boxedminipage}{\funcwidth}

    \raggedright \textbf{isValid}(\textit{self})

    \vspace{-1.5ex}

    \rule{\textwidth}{0.5\fboxrule}
\setlength{\parskip}{2ex}
    Test if the graph in the map class is valid or not.

\setlength{\parskip}{1ex}
      \textbf{Return Value}
    \vspace{-1ex}

      \begin{quote}
      graph is valid or not

      {\it (type=boolean)}

      \end{quote}

    \end{boxedminipage}

    \label{UnBlockMeSolver:Map:Map:Map:validAdditionMoves}
    \index{UnBlockMeSolver \textit{(package)}!UnBlockMeSolver.Map \textit{(package)}!UnBlockMeSolver.Map.Map \textit{(module)}!UnBlockMeSolver.Map.Map.Map \textit{(class)}!UnBlockMeSolver.Map.Map.Map.validAdditionMoves \textit{(method)}}

    \vspace{0.5ex}

\hspace{.8\funcindent}\begin{boxedminipage}{\funcwidth}

    \raggedright \textbf{validAdditionMoves}(\textit{self}, \textit{move})

    \vspace{-1.5ex}

    \rule{\textwidth}{0.5\fboxrule}
\setlength{\parskip}{2ex}
    This will handle move verificatin for moving to the right or down in 
    the board. It will return an array of the the moves that can be made in
    the direction given.

\setlength{\parskip}{1ex}
      \textbf{Parameters}
      \vspace{-1ex}

      \begin{quote}
        \begin{Ventry}{xxxx}

          \item[move]

          Array of moves in the direction of the move given

            {\it (type=[Move])}

        \end{Ventry}

      \end{quote}

    \end{boxedminipage}

    \label{UnBlockMeSolver:Map:Map:Map:validSubtractionMoves}
    \index{UnBlockMeSolver \textit{(package)}!UnBlockMeSolver.Map \textit{(package)}!UnBlockMeSolver.Map.Map \textit{(module)}!UnBlockMeSolver.Map.Map.Map \textit{(class)}!UnBlockMeSolver.Map.Map.Map.validSubtractionMoves \textit{(method)}}

    \vspace{0.5ex}

\hspace{.8\funcindent}\begin{boxedminipage}{\funcwidth}

    \raggedright \textbf{validSubtractionMoves}(\textit{self}, \textit{move})

    \vspace{-1.5ex}

    \rule{\textwidth}{0.5\fboxrule}
\setlength{\parskip}{2ex}
    This will handle move verificatin for moving to the left or up in the 
    board. It will return an array of the the moves that can be made in the
    direction given.

\setlength{\parskip}{1ex}
      \textbf{Parameters}
      \vspace{-1ex}

      \begin{quote}
        \begin{Ventry}{xxxx}

          \item[move]

          Array of moves in the direction of the move given

            {\it (type=[Move])}

        \end{Ventry}

      \end{quote}

    \end{boxedminipage}

    \label{UnBlockMeSolver:Map:Map:Map:isValidMove}
    \index{UnBlockMeSolver \textit{(package)}!UnBlockMeSolver.Map \textit{(package)}!UnBlockMeSolver.Map.Map \textit{(module)}!UnBlockMeSolver.Map.Map.Map \textit{(class)}!UnBlockMeSolver.Map.Map.Map.isValidMove \textit{(method)}}

    \vspace{0.5ex}

\hspace{.8\funcindent}\begin{boxedminipage}{\funcwidth}

    \raggedright \textbf{isValidMove}(\textit{self}, \textit{move})

    \vspace{-1.5ex}

    \rule{\textwidth}{0.5\fboxrule}
\setlength{\parskip}{2ex}
    Check if the move given is valid or not to make the move.

\setlength{\parskip}{1ex}
      \textbf{Return Value}
    \vspace{-1ex}

      \begin{quote}
      if the move is valid or not

      {\it (type=boolean)}

      \end{quote}

    \end{boxedminipage}

    \label{UnBlockMeSolver:Map:Map:Map:makeConfidentMove}
    \index{UnBlockMeSolver \textit{(package)}!UnBlockMeSolver.Map \textit{(package)}!UnBlockMeSolver.Map.Map \textit{(module)}!UnBlockMeSolver.Map.Map.Map \textit{(class)}!UnBlockMeSolver.Map.Map.Map.makeConfidentMove \textit{(method)}}

    \vspace{0.5ex}

\hspace{.8\funcindent}\begin{boxedminipage}{\funcwidth}

    \raggedright \textbf{makeConfidentMove}(\textit{self}, \textit{move})

    \vspace{-1.5ex}

    \rule{\textwidth}{0.5\fboxrule}
\setlength{\parskip}{2ex}
    Make a move on the board without checking for its validity.

\setlength{\parskip}{1ex}
      \textbf{Parameters}
      \vspace{-1ex}

      \begin{quote}
        \begin{Ventry}{xxxx}

          \item[move]

          move to make on the board

            {\it (type=Move)}

        \end{Ventry}

      \end{quote}

    \end{boxedminipage}

    \label{UnBlockMeSolver:Map:Map:Map:makeMove}
    \index{UnBlockMeSolver \textit{(package)}!UnBlockMeSolver.Map \textit{(package)}!UnBlockMeSolver.Map.Map \textit{(module)}!UnBlockMeSolver.Map.Map.Map \textit{(class)}!UnBlockMeSolver.Map.Map.Map.makeMove \textit{(method)}}

    \vspace{0.5ex}

\hspace{.8\funcindent}\begin{boxedminipage}{\funcwidth}

    \raggedright \textbf{makeMove}(\textit{self}, \textit{move})

    \vspace{-1.5ex}

    \rule{\textwidth}{0.5\fboxrule}
\setlength{\parskip}{2ex}
    Make a move on the board. Raise syntax error on bad move given

\setlength{\parskip}{1ex}
      \textbf{Parameters}
      \vspace{-1ex}

      \begin{quote}
        \begin{Ventry}{xxxx}

          \item[move]

          move to make on the board

            {\it (type=Move)}

        \end{Ventry}

      \end{quote}

    \end{boxedminipage}

    \label{UnBlockMeSolver:Map:Map:Map:getMoves}
    \index{UnBlockMeSolver \textit{(package)}!UnBlockMeSolver.Map \textit{(package)}!UnBlockMeSolver.Map.Map \textit{(module)}!UnBlockMeSolver.Map.Map.Map \textit{(class)}!UnBlockMeSolver.Map.Map.Map.getMoves \textit{(method)}}

    \vspace{0.5ex}

\hspace{.8\funcindent}\begin{boxedminipage}{\funcwidth}

    \raggedright \textbf{getMoves}(\textit{self})

    \vspace{-1.5ex}

    \rule{\textwidth}{0.5\fboxrule}
\setlength{\parskip}{2ex}
    Get the moves available on the board.

\setlength{\parskip}{1ex}
      \textbf{Return Value}
    \vspace{-1ex}

      \begin{quote}
      Array of moves available

      {\it (type=[Move])}

      \end{quote}

    \end{boxedminipage}

    \label{UnBlockMeSolver:Map:Map:Map:isSolved}
    \index{UnBlockMeSolver \textit{(package)}!UnBlockMeSolver.Map \textit{(package)}!UnBlockMeSolver.Map.Map \textit{(module)}!UnBlockMeSolver.Map.Map.Map \textit{(class)}!UnBlockMeSolver.Map.Map.Map.isSolved \textit{(method)}}

    \vspace{0.5ex}

\hspace{.8\funcindent}\begin{boxedminipage}{\funcwidth}

    \raggedright \textbf{isSolved}(\textit{self})

    \vspace{-1.5ex}

    \rule{\textwidth}{0.5\fboxrule}
\setlength{\parskip}{2ex}
    If there is a goal found then the board is not solved. Complete 
    checking is done as there may be a goal in any location.

\setlength{\parskip}{1ex}
      \textbf{Return Value}
    \vspace{-1ex}

      \begin{quote}
      if the game has been solved or not

      {\it (type=boolean)}

      \end{quote}

    \end{boxedminipage}

    \label{UnBlockMeSolver:Map:Map:Map:copy}
    \index{UnBlockMeSolver \textit{(package)}!UnBlockMeSolver.Map \textit{(package)}!UnBlockMeSolver.Map.Map \textit{(module)}!UnBlockMeSolver.Map.Map.Map \textit{(class)}!UnBlockMeSolver.Map.Map.Map.copy \textit{(method)}}

    \vspace{0.5ex}

\hspace{.8\funcindent}\begin{boxedminipage}{\funcwidth}

    \raggedright \textbf{copy}(\textit{self})

    \vspace{-1.5ex}

    \rule{\textwidth}{0.5\fboxrule}
\setlength{\parskip}{2ex}
    Create a copy of this map and return it

\setlength{\parskip}{1ex}
      \textbf{Return Value}
    \vspace{-1ex}

      \begin{quote}
      Copy of this map

      {\it (type=Map)}

      \end{quote}

    \end{boxedminipage}

    \label{UnBlockMeSolver:Map:Map:Map:copyMove}
    \index{UnBlockMeSolver \textit{(package)}!UnBlockMeSolver.Map \textit{(package)}!UnBlockMeSolver.Map.Map \textit{(module)}!UnBlockMeSolver.Map.Map.Map \textit{(class)}!UnBlockMeSolver.Map.Map.Map.copyMove \textit{(method)}}

    \vspace{0.5ex}

\hspace{.8\funcindent}\begin{boxedminipage}{\funcwidth}

    \raggedright \textbf{copyMove}(\textit{self}, \textit{move})

    \vspace{-1.5ex}

    \rule{\textwidth}{0.5\fboxrule}
\setlength{\parskip}{2ex}
    Create a copy of the map and make a move on it; return the result.

\setlength{\parskip}{1ex}
      \textbf{Parameters}
      \vspace{-1ex}

      \begin{quote}
        \begin{Ventry}{xxxx}

          \item[move]

          move to be bade on copied board

            {\it (type=Move)}

        \end{Ventry}

      \end{quote}

      \textbf{Return Value}
    \vspace{-1ex}

      \begin{quote}
      The new map with the move made on it

      {\it (type=Map)}

      \end{quote}

    \end{boxedminipage}

    \label{UnBlockMeSolver:Map:Map:Map:copyConfidentMove}
    \index{UnBlockMeSolver \textit{(package)}!UnBlockMeSolver.Map \textit{(package)}!UnBlockMeSolver.Map.Map \textit{(module)}!UnBlockMeSolver.Map.Map.Map \textit{(class)}!UnBlockMeSolver.Map.Map.Map.copyConfidentMove \textit{(method)}}

    \vspace{0.5ex}

\hspace{.8\funcindent}\begin{boxedminipage}{\funcwidth}

    \raggedright \textbf{copyConfidentMove}(\textit{self}, \textit{move})

    \vspace{-1.5ex}

    \rule{\textwidth}{0.5\fboxrule}
\setlength{\parskip}{2ex}
    Create a copy of the map and make a move on it without checking for its
    validity. Return the result.

\setlength{\parskip}{1ex}
      \textbf{Parameters}
      \vspace{-1ex}

      \begin{quote}
        \begin{Ventry}{xxxx}

          \item[move]

          move to be bade on copied board

            {\it (type=Move)}

        \end{Ventry}

      \end{quote}

      \textbf{Return Value}
    \vspace{-1ex}

      \begin{quote}
      The new map with the move made on it

      {\it (type=Map)}

      \end{quote}

    \end{boxedminipage}


\large{\textbf{\textit{Inherited from object}}}

\begin{quote}
\_\_delattr\_\_(), \_\_format\_\_(), \_\_getattribute\_\_(), \_\_hash\_\_(), \_\_new\_\_(), \_\_reduce\_\_(), \_\_reduce\_ex\_\_(), \_\_repr\_\_(), \_\_setattr\_\_(), \_\_sizeof\_\_(), \_\_str\_\_(), \_\_subclasshook\_\_()
\end{quote}

%%%%%%%%%%%%%%%%%%%%%%%%%%%%%%%%%%%%%%%%%%%%%%%%%%%%%%%%%%%%%%%%%%%%%%%%%%%
%%                              Properties                               %%
%%%%%%%%%%%%%%%%%%%%%%%%%%%%%%%%%%%%%%%%%%%%%%%%%%%%%%%%%%%%%%%%%%%%%%%%%%%

  \subsubsection{Properties}

    \vspace{-1cm}
\hspace{\varindent}\begin{longtable}{|p{\varnamewidth}|p{\vardescrwidth}|l}
\cline{1-2}
\cline{1-2} \centering \textbf{Name} & \centering \textbf{Description}& \\
\cline{1-2}
\endhead\cline{1-2}\multicolumn{3}{r}{\small\textit{continued on next page}}\\\endfoot\cline{1-2}
\endlastfoot\multicolumn{2}{|l|}{\textit{Inherited from object}}\\
\multicolumn{2}{|p{\varwidth}|}{\raggedright \_\_class\_\_}\\
\cline{1-2}
\end{longtable}


%%%%%%%%%%%%%%%%%%%%%%%%%%%%%%%%%%%%%%%%%%%%%%%%%%%%%%%%%%%%%%%%%%%%%%%%%%%
%%                            Class Variables                            %%
%%%%%%%%%%%%%%%%%%%%%%%%%%%%%%%%%%%%%%%%%%%%%%%%%%%%%%%%%%%%%%%%%%%%%%%%%%%

  \subsubsection{Class Variables}

    \vspace{-1cm}
\hspace{\varindent}\begin{longtable}{|p{\varnamewidth}|p{\vardescrwidth}|l}
\cline{1-2}
\cline{1-2} \centering \textbf{Name} & \centering \textbf{Description}& \\
\cline{1-2}
\endhead\cline{1-2}\multicolumn{3}{r}{\small\textit{continued on next page}}\\\endfoot\cline{1-2}
\endlastfoot\raggedright w\-a\-l\-l\- & \raggedright \textbf{Value:} 
{\tt \texttt{'}\texttt{{\textbar}}\texttt{'}}&\\
\cline{1-2}
\raggedright g\-o\-a\-l\- & \raggedright \textbf{Value:} 
{\tt \texttt{'}\texttt{\$}\texttt{'}}&\\
\cline{1-2}
\raggedright e\-m\-p\-t\-y\- & \raggedright \textbf{Value:} 
{\tt \texttt{'}\texttt{0}\texttt{'}}&\\
\cline{1-2}
\raggedright p\-l\-a\-y\-e\-r\- & \raggedright \textbf{Value:} 
{\tt \texttt{'}\texttt{**}\texttt{'}}&\\
\cline{1-2}
\raggedright p\-l\-a\-y\-e\-r\-P\-i\-e\-c\-e\- & \raggedright \textbf{Value:} 
{\tt \texttt{'}\texttt{*}\texttt{'}}&\\
\cline{1-2}
\end{longtable}

    \index{UnBlockMeSolver \textit{(package)}!UnBlockMeSolver.Map \textit{(package)}!UnBlockMeSolver.Map.Map \textit{(module)}!UnBlockMeSolver.Map.Map.Map \textit{(class)}|)}
    \index{UnBlockMeSolver \textit{(package)}!UnBlockMeSolver.Map \textit{(package)}!UnBlockMeSolver.Map.Map \textit{(module)}|)}
